% ***********************************************
%***********COSAS A MEJORAR O CORREGIR***********
%
%
%
%
%
%
%
% ***********************************************
% Preamble
% ---
\documentclass[10pt,a4paper]{article}

% Packages
% ---
\usepackage{float}
\usepackage[utf8]{inputenc}
\usepackage{amsmath}
\usepackage{mathtools}
\usepackage{amsfonts}
\usepackage{amssymb}
\usepackage{layout}
\usepackage{graphics}
\usepackage{graphicx}
\usepackage[spanish]{babel}
\usepackage{lastpage}
\usepackage{tabto}
\usepackage{xcolor}
\usepackage[a4paper, top=1 in,bottom=1in, left=1 in, right=0.7 in]{geometry}
\usepackage{fancyhdr}
\pagestyle{fancy}
\usepackage{comment}
\usepackage{svg}




\lhead{\footnotesize Teoría de los circuitos 1 - Año \the\year{} Ver. 2022.01.12}
\chead{}
\rhead{\footnotesize Trabajo Práctico N$^\text o$6: Sistemas polifásicos.}
\lfoot{}
\cfoot{}
\rfoot{\footnotesize P\'agina \thepage\ de \pageref{LastPage}}
\renewcommand{\headrulewidth}{0.4pt}
\renewcommand{\footrulewidth}{0.4pt}

\title{
	\textsc{\includegraphics[width=0.35\textwidth]{logoUTN.jpg}} ~\\
	{\large Departamento de Electr\'onica}\\ 
	[0.1cm]
	{\Huge{Teoría de circuitos 1}} \\
	[0.25cm]
	{\Large{Trabajo Práctico N$^{\text {o}}$6: Sistemas polifásicos}		\\
}}
\author{}
\date{}
    
\begin{document}
	\maketitle
	

\begin{enumerate}
	
% ***********************************************
% 					Ejercicio 1
% ***********************************************

\item Un sistema trifásico de tres conductores, alimenta una carga en triángulo compuesta por tres impedancias iguales. Determinar las corrientes de fase y de línea, y trazar el diagrama fasorial. Donde $V_{AC}=150\ /240^{\circ}V$ $V_{CB}=150\ /0^{\circ}V$ $V_{BA}=150\ /120^{\circ}V$ $Z_c=25/60^{\circ}\Omega$

\begin{figure}[H]
	\centering
	\includegraphics[scale=0.8]{img/1.png}
	\label{e1}
\end{figure}


% ***********************************************
% 					Ejercicio 2
% ***********************************************
\item Dado el circuito de la figura encontrar el valor de $I_0$ y determinar si se trata de un circuito trifásico equilibrado o desequilibrado. Justificar su respuesta.

$V_1=120V \angle 0^{\circ}$;
$V_2=120V \angle -120^{\circ}$;
$V_3=120V \angle 120^{\circ}$.

\begin{figure}[H]
	\centering
	\includegraphics[scale=0.4]{img/2.png}
	\label{e2}
\end{figure}

% ***********************************************
% 					Ejercicio 3
% ***********************************************
\item Se conectan en triángulo tres impedancias iguales de $25/76^{\circ}\Omega$ a un sistema trifásico de tres conductores de $125 V$ y secuenca ABC. Hallar las intencidades de corriente de línea.


% ***********************************************
% 					Ejercicio 4
% ***********************************************
\item Una carga equilibrada con impedancias de $65/-20^{\circ}\Omega$ se conecta en estrella a un sistema trifásico de tres conductores, $480V$ y secuencia CBA. Hallar las intensidades de corriente de línea y la potencia total.



% ***********************************************
% 					Ejercicio 5
% ***********************************************
\item Tres impedancias idénticas de $9/-30^{\circ}\Omega$ en triángulo y tres impedancias de $5/45^{\circ}\Omega$ en estrella se conectan al mismo sistema trifásico de $3$ conductores, $480V$ y secuencia ABC. Hallar el módulo de la corriente de línea.

% ***********************************************
% 					Ejercicio 6
% ***********************************************
\item La salida de la fuente trifásica, equilibrada y de secuencia ABC que se muestra en la figura, es de $60KVA$ con un factor de potencia en retardo de $0,96$. El voltaje de línea en la fuente es $680V$. Encontrar el voltaje de línea en la carga y la potencia compleja total en los terminales de la carga.

\begin{figure}[H]
	\centering
	\includegraphics[scale=0.4]{img/3.png}
	\label{e6}
\end{figure}
% ***********************************************
% 					Ejercicio  7
% ***********************************************
\item Tres cargas inductivas desiguales se conectan entre las fases **a-b**, **b-c** y **c-a** de un sistema trifásico. Los voltajes entre fases son iguales todos a $230 V$, estando $V_{ab}$ adelantado de $V_{bc}$. La primera carga toma $100A$ conn un factor de potencia de $0,707$, la segunda toma $150A$ con un factor de potencia de $0,8$ y la tercera toma $28,4KW$ con un factor de potencia de $0,5$. 
\begin{enumerate}
	\item Calcular loos  valores de las corrientes de línea.
	\item ¿Qué indicarían dos vatímetros conectados según el método de los dos vatímetros, si sus bobinas de corriente están en las fases a y b.
\end{enumerate}

\section*{Referencias}
Secuencia positiva o ABC:

$V_{AN}= \frac{V_L}{\sqrt{3}}\ \angle0^{\circ}  $    \\
$V_{BN}= \frac{V_L}{\sqrt{3}}\ \angle-120^{\circ}  $    \\
$V_{CN}= \frac{V_L}{\sqrt{3}}\ \angle120^{\circ}  $    \\
$V_{AB} = V_L\ \angle 30^{\circ} $    \\
$V_{BC} = V_L\ \angle -90^{\circ}  $    \\
$V_{CA} = V_L\ \angle 150^{\circ} $    \\

Secuencia negativa o CBA

$V_{AN}= \frac{V_L}{\sqrt{3}}\ \angle 0^{\circ}  $    \\
$V_{BN}= \frac{V_L}{\sqrt{3}}\ \angle 120^{\circ}  $    \\
$V_{CN}= \frac{V_L}{\sqrt{3}}\ \angle -120^{\circ}  $  \\
$V_{AC} = V_L\ \angle 30^{\circ} $    \\
$V_{CB} = V_L\ \angle -90^{\circ}  $    \\
$V_{BA} = V_L\ \angle 150^{\circ} $    \\



\begin{figure}[H]
	\centering
	\includegraphics[scale=0.5]{img/secp.png}
	\caption{(a) Secuencia Positiva. b) Secuencuencia Negativa}
	\label{sec}
\end{figure}




    
\end{enumerate}
    
\end{document}
