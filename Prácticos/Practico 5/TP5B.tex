% ***********************************************
%***********COSAS A MEJORAR O CORREGIR***********
%
%
%
%
%
%
%
% ***********************************************
% Preamble
% ---
\documentclass[10pt,a4paper]{article}

% Packages
% ---
\usepackage{float}
\usepackage[utf8]{inputenc}
\usepackage{amsmath}
\usepackage{mathtools}
\usepackage{amsfonts}
\usepackage{amssymb}
\usepackage{layout}
\usepackage{graphics}
\usepackage{graphicx}
\usepackage[spanish]{babel}
\usepackage{lastpage}
\usepackage{tabto}
\usepackage{xcolor}
\usepackage[a4paper, top=1 in,bottom=1in, left=1 in, right=0.7 in]{geometry}
\usepackage{fancyhdr}
\pagestyle{fancy}
\usepackage{comment}





\lhead{\footnotesize Teoría de los circuitos 1 - Año \the\year{} Ver. 2022.01.12}
\chead{}
\rhead{\footnotesize Trabajo Práctico N$^\text o$5.}
\lfoot{}
\cfoot{}
\rfoot{\footnotesize P\'agina \thepage\ de \pageref{LastPage}}
\renewcommand{\headrulewidth}{0.4pt}
\renewcommand{\footrulewidth}{0.4pt}

\title{
	\textsc{\includegraphics[width=0.35\textwidth]{logoUTN.jpg}} ~\\
	{\large Departamento de Electr\'onica}\\ 
	[0.1cm]
	{\Huge{Teoría de circuitos 1}} \\
	[0.25cm]
	{\Large{Trabajo Práctico N$^{\text {o}}$5 parte B: Potencia en CC y CA}		
}}
\author{}
\date{}
    
\begin{document}
	\maketitle

\begin{enumerate}
	
% ***********************************************
% 					Ejercicio 1
% ***********************************************

\item en el siguiente circuito se indican las potencias consumidas por cada resistencia, determinar:

\begin{enumerate}
	\item La disipación de potencia total
	\item La potencia reactiva total
	\item la corriente de la fuente $I_g$
	\item El valor de cada resistencia 
	\item Determine las corrientes $I_1$ e $I_2$
\end{enumerate}

\begin{figure}[H]
	\centering
	\includegraphics[scale=0.9]{img/1.png}
	\label{e1}
\end{figure}

Rta: a. $P_T=130$ W, b. $Q_T=0$ VAR, $S_T=130$ VA, c. $I_g=0.542$ A, d. $R_1=204.5\Omega$, $R_2=371.12\Omega$, $R_3=668.9\Omega$, e. $I_2=0.384A$, $I_3=0.193A$


% ***********************************************
% 					Ejercicio 2
% ***********************************************
\item Para el sistema de la figura: 

\begin{enumerate}
	\item Determinar la potencia activa total, la potencia reactiva total, la potencia aparente total y el factor de potencia  
	\item Trace el triángulo de potencias   
	\item Determine la corriente $I_f$
\end{enumerate}

\begin{figure}[H]
	\centering
	\includegraphics[scale=0.6]{img/2.png}
	\label{e2}
\end{figure}

% ***********************************************
% 					Ejercicio 3
% ***********************************************
\item Dado el siguiente circuito deterimne:   

\begin{enumerate}
	\item La potencia activa suministrada a cada elemento   
	\item La potencia reactiva en cada elemento
	\item La potencia aparente en cada elemento   
	\item Determine $P_T$, $Q_T$, $S_T$ y $F_p$   
	\item Grafique el trángulo de potencias  
	\item Determine $I_t$    
\end{enumerate}
\begin{figure}[H]
	\centering
	\includegraphics[scale=0.8]{img/3.png}
	\label{e3}
\end{figure}

Rta: a. $P_R=20W$ $P_{L,C}=0W$, b. $Q_R=0$VAR $Q_C=80$ VAR $ Q_L=100$ VAR c. $S_R=200$ VA $S_L=j100$VA $S_C=j80VA$ d. $P_T=200$ W  $Q_T=20$VAR ($L$) $S_T= (200+j20)$VA $F_p=0.995$ e. $I_t=10.05A \angle-5.73^\circ$

% ***********************************************
% 					Ejercicio 4
% ***********************************************

\item  Utilizando el teorema de Thévenin, hallar la potencia disipada en una resistencia de $1\Omega$, conectada en los terminales $A-B$ del circuito de la figura.
\begin{figure}[H]
	\centering
	\includegraphics[scale=0.8]{img/4.png}
	\label{e4}
\end{figure}

Rta: $P=1.5$W

% ***********************************************
% 					Ejercicio 5
% ***********************************************
\item Una carga eléctrica funciona a $240$ V (rms). La carga absorbe una potencia media de $8$ KW
con un factor de potencia de $0.8$ en atrazo. Calcular la impedancia y la potencia compleja de
la carga.  
Rta: $Z=(4.608+j3.456)\Omega$ $S=(8+j6)$ KVA

% ***********************************************
% 					Ejercicio 6
% ***********************************************
 \item En el circuito de la figura, una carga con impedancia de $(39 + j 26)\Omega$ está conectada a una fuente de voltaje a través de una línea que tiene impedancia de $(1 +j4)\Omega$. El valor eficaz de la fuente de voltaje es de $250$V. Calcular:   
 \begin{enumerate}
 	\item La potencia media y la reactiva que se suministran a la carga   
 	\item La potencia media y la reactiva que se suministran a la línea   
 	\item La potencia media y la reactiva que suministra la fuente. 	
 \end{enumerate}
	
\begin{figure}[H]
	\centering
	\includegraphics[scale=0.8]{img/5.png}
	\label{e6}
\end{figure}
  
Rta: a. $S=(975+j650)$VA b. $P=25W$ $Q=100$VA c. $S=(1000+j750)$VA
% ***********************************************
% 					Ejercicio  7
% ***********************************************
\item El circuito serie de la figura consume $300 W$ con un factor de potencia de $0,6$ en retraso. Hallar el valor de la impedancia y determinar el triángulo de potencias:.
\begin{figure}[H]
	\centering
	\includegraphics[scale=0.8]{img/6.png}
	\label{e7}
\end{figure}

Rta: $Z=j4\Omega$, $S=(300+j400)$VA
% ***********************************************
% 					Ejercicio 8
% ***********************************************
\item Una instalación consume una potencia activa de $5,2$ kW y una potencia reactiva de $1,1$ kVAR en atraso. Calcular el ángulo de desfasaje y el factor de potencia.
% ***********************************************
% 					Ejercicio 9
% ***********************************************
\item Una instalación consume $3,5kW $de potencia activa y tiene un factor de potencia de $0,8$. Calcular la potencia reactiva y la potencia aparente.


% ***********************************************
% 					Ejercicio 10
% ***********************************************

\item Una instalación alimentada con $220$V  y $50$Hz consume una potencia activa de $2$ kW con factor de potencia de $0.8$ y corriente en atraso. Calcular la capacidad necesaria a conectar en paralelo a la línea para obtener un factor de potencia de $0,95$.


% ***********************************************
% 					Ejercicio 11
% ***********************************************
\item Un motor monofásico que consume una potencia aparente $S=300$VA y tiene un factor de potencia $FP= 0,7$ se conecta al sistema de distribución nacional ($V_{ef}=220$V y $f=50$Hz).
\begin{enumerate}
	\item Calcular el valor del capacitor para llevar el factor de potencia a $0,93$ en atraso.
	\item Calcular la corriente entregada al sistema antes y después de la corrección. 
\end{enumerate}




% ***********************************************
% 					Ejercicio 12
% ***********************************************
\item Demostrar el teorema de máxima transferencia de potencia.


% ***********************************************
% 					Ejercicio 13
% ***********************************************


% ***********************************************
% 					Ejercicio 14
% ***********************************************

% ***********************************************
% 					Ejercicio 15
% ***********************************************



% ***********************************************
% 					Ejercicio 16
% ***********************************************


% ***********************************************
% 					Ejercicio 17
% ***********************************************



% ***********************************************
% 					Ejercicio  18
% ***********************************************



    
\end{enumerate}
    
\end{document}
