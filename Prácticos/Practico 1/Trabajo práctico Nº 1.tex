\documentclass[11pt]{article}

    \usepackage[breakable]{tcolorbox}
    \usepackage{parskip} % Stop auto-indenting (to mimic markdown behaviour)
    
    \usepackage{iftex}
    \ifPDFTeX
    	\usepackage[T1]{fontenc}
    	\usepackage{mathpazo}
    \else
    	\usepackage{fontspec}
    \fi

    % Basic figure setup, for now with no caption control since it's done
    % automatically by Pandoc (which extracts ![](path) syntax from Markdown).
    \usepackage{graphicx}
    % Maintain compatibility with old templates. Remove in nbconvert 6.0
    \let\Oldincludegraphics\includegraphics
    % Ensure that by default, figures have no caption (until we provide a
    % proper Figure object with a Caption API and a way to capture that
    % in the conversion process - todo).
    \usepackage{caption}
    \DeclareCaptionFormat{nocaption}{}
    \captionsetup{format=nocaption,aboveskip=0pt,belowskip=0pt}

    \usepackage[Export]{adjustbox} % Used to constrain images to a maximum size
    \adjustboxset{max size={0.9\linewidth}{0.9\paperheight}}
    \usepackage{float}
    \floatplacement{figure}{H} % forces figures to be placed at the correct location
    \usepackage{xcolor} % Allow colors to be defined
    \usepackage{enumerate} % Needed for markdown enumerations to work
    \usepackage{geometry} % Used to adjust the document margins
    \usepackage{amsmath} % Equations
    \usepackage{amssymb} % Equations
    \usepackage{textcomp} % defines textquotesingle
    % Hack from http://tex.stackexchange.com/a/47451/13684:
    \AtBeginDocument{%
        \def\PYZsq{\textquotesingle}% Upright quotes in Pygmentized code
    }
    \usepackage{upquote} % Upright quotes for verbatim code
    \usepackage{eurosym} % defines \euro
    \usepackage[mathletters]{ucs} % Extended unicode (utf-8) support
    \usepackage{fancyvrb} % verbatim replacement that allows latex
    \usepackage{grffile} % extends the file name processing of package graphics 
                         % to support a larger range
    \makeatletter % fix for grffile with XeLaTeX
    \def\Gread@@xetex#1{%
      \IfFileExists{"\Gin@base".bb}%
      {\Gread@eps{\Gin@base.bb}}%
      {\Gread@@xetex@aux#1}%
    }
    \makeatother

    % The hyperref package gives us a pdf with properly built
    % internal navigation ('pdf bookmarks' for the table of contents,
    % internal cross-reference links, web links for URLs, etc.)
    \usepackage{hyperref}
    % The default LaTeX title has an obnoxious amount of whitespace. By default,
    % titling removes some of it. It also provides customization options.
    \usepackage{titling}
    \usepackage{longtable} % longtable support required by pandoc >1.10
    \usepackage{booktabs}  % table support for pandoc > 1.12.2
    \usepackage[inline]{enumitem} % IRkernel/repr support (it uses the enumerate* environment)
    \usepackage[normalem]{ulem} % ulem is needed to support strikethroughs (\sout)
                                % normalem makes italics be italics, not underlines
    \usepackage{mathrsfs}
    

    
    % Colors for the hyperref package
    \definecolor{urlcolor}{rgb}{0,.145,.698}
    \definecolor{linkcolor}{rgb}{.71,0.21,0.01}
    \definecolor{citecolor}{rgb}{.12,.54,.11}

    % ANSI colors
    \definecolor{ansi-black}{HTML}{3E424D}
    \definecolor{ansi-black-intense}{HTML}{282C36}
    \definecolor{ansi-red}{HTML}{E75C58}
    \definecolor{ansi-red-intense}{HTML}{B22B31}
    \definecolor{ansi-green}{HTML}{00A250}
    \definecolor{ansi-green-intense}{HTML}{007427}
    \definecolor{ansi-yellow}{HTML}{DDB62B}
    \definecolor{ansi-yellow-intense}{HTML}{B27D12}
    \definecolor{ansi-blue}{HTML}{208FFB}
    \definecolor{ansi-blue-intense}{HTML}{0065CA}
    \definecolor{ansi-magenta}{HTML}{D160C4}
    \definecolor{ansi-magenta-intense}{HTML}{A03196}
    \definecolor{ansi-cyan}{HTML}{60C6C8}
    \definecolor{ansi-cyan-intense}{HTML}{258F8F}
    \definecolor{ansi-white}{HTML}{C5C1B4}
    \definecolor{ansi-white-intense}{HTML}{A1A6B2}
    \definecolor{ansi-default-inverse-fg}{HTML}{FFFFFF}
    \definecolor{ansi-default-inverse-bg}{HTML}{000000}

    % commands and environments needed by pandoc snippets
    % extracted from the output of `pandoc -s`
    \providecommand{\tightlist}{%
      \setlength{\itemsep}{0pt}\setlength{\parskip}{0pt}}
    \DefineVerbatimEnvironment{Highlighting}{Verbatim}{commandchars=\\\{\}}
    % Add ',fontsize=\small' for more characters per line
    \newenvironment{Shaded}{}{}
    \newcommand{\KeywordTok}[1]{\textcolor[rgb]{0.00,0.44,0.13}{\textbf{{#1}}}}
    \newcommand{\DataTypeTok}[1]{\textcolor[rgb]{0.56,0.13,0.00}{{#1}}}
    \newcommand{\DecValTok}[1]{\textcolor[rgb]{0.25,0.63,0.44}{{#1}}}
    \newcommand{\BaseNTok}[1]{\textcolor[rgb]{0.25,0.63,0.44}{{#1}}}
    \newcommand{\FloatTok}[1]{\textcolor[rgb]{0.25,0.63,0.44}{{#1}}}
    \newcommand{\CharTok}[1]{\textcolor[rgb]{0.25,0.44,0.63}{{#1}}}
    \newcommand{\StringTok}[1]{\textcolor[rgb]{0.25,0.44,0.63}{{#1}}}
    \newcommand{\CommentTok}[1]{\textcolor[rgb]{0.38,0.63,0.69}{\textit{{#1}}}}
    \newcommand{\OtherTok}[1]{\textcolor[rgb]{0.00,0.44,0.13}{{#1}}}
    \newcommand{\AlertTok}[1]{\textcolor[rgb]{1.00,0.00,0.00}{\textbf{{#1}}}}
    \newcommand{\FunctionTok}[1]{\textcolor[rgb]{0.02,0.16,0.49}{{#1}}}
    \newcommand{\RegionMarkerTok}[1]{{#1}}
    \newcommand{\ErrorTok}[1]{\textcolor[rgb]{1.00,0.00,0.00}{\textbf{{#1}}}}
    \newcommand{\NormalTok}[1]{{#1}}
    
    % Additional commands for more recent versions of Pandoc
    \newcommand{\ConstantTok}[1]{\textcolor[rgb]{0.53,0.00,0.00}{{#1}}}
    \newcommand{\SpecialCharTok}[1]{\textcolor[rgb]{0.25,0.44,0.63}{{#1}}}
    \newcommand{\VerbatimStringTok}[1]{\textcolor[rgb]{0.25,0.44,0.63}{{#1}}}
    \newcommand{\SpecialStringTok}[1]{\textcolor[rgb]{0.73,0.40,0.53}{{#1}}}
    \newcommand{\ImportTok}[1]{{#1}}
    \newcommand{\DocumentationTok}[1]{\textcolor[rgb]{0.73,0.13,0.13}{\textit{{#1}}}}
    \newcommand{\AnnotationTok}[1]{\textcolor[rgb]{0.38,0.63,0.69}{\textbf{\textit{{#1}}}}}
    \newcommand{\CommentVarTok}[1]{\textcolor[rgb]{0.38,0.63,0.69}{\textbf{\textit{{#1}}}}}
    \newcommand{\VariableTok}[1]{\textcolor[rgb]{0.10,0.09,0.49}{{#1}}}
    \newcommand{\ControlFlowTok}[1]{\textcolor[rgb]{0.00,0.44,0.13}{\textbf{{#1}}}}
    \newcommand{\OperatorTok}[1]{\textcolor[rgb]{0.40,0.40,0.40}{{#1}}}
    \newcommand{\BuiltInTok}[1]{{#1}}
    \newcommand{\ExtensionTok}[1]{{#1}}
    \newcommand{\PreprocessorTok}[1]{\textcolor[rgb]{0.74,0.48,0.00}{{#1}}}
    \newcommand{\AttributeTok}[1]{\textcolor[rgb]{0.49,0.56,0.16}{{#1}}}
    \newcommand{\InformationTok}[1]{\textcolor[rgb]{0.38,0.63,0.69}{\textbf{\textit{{#1}}}}}
    \newcommand{\WarningTok}[1]{\textcolor[rgb]{0.38,0.63,0.69}{\textbf{\textit{{#1}}}}}
    
    
    % Define a nice break command that doesn't care if a line doesn't already
    % exist.
    \def\br{\hspace*{\fill} \\* }
    % Math Jax compatibility definitions
    \def\gt{>}
    \def\lt{<}
    \let\Oldtex\TeX
    \let\Oldlatex\LaTeX
    \renewcommand{\TeX}{\textrm{\Oldtex}}
    \renewcommand{\LaTeX}{\textrm{\Oldlatex}}
    % Document parameters
    % Document title
    \title{Trabajo práctico Nº 1}
    
    
    
    
    
% Pygments definitions
\makeatletter
\def\PY@reset{\let\PY@it=\relax \let\PY@bf=\relax%
    \let\PY@ul=\relax \let\PY@tc=\relax%
    \let\PY@bc=\relax \let\PY@ff=\relax}
\def\PY@tok#1{\csname PY@tok@#1\endcsname}
\def\PY@toks#1+{\ifx\relax#1\empty\else%
    \PY@tok{#1}\expandafter\PY@toks\fi}
\def\PY@do#1{\PY@bc{\PY@tc{\PY@ul{%
    \PY@it{\PY@bf{\PY@ff{#1}}}}}}}
\def\PY#1#2{\PY@reset\PY@toks#1+\relax+\PY@do{#2}}

\expandafter\def\csname PY@tok@w\endcsname{\def\PY@tc##1{\textcolor[rgb]{0.73,0.73,0.73}{##1}}}
\expandafter\def\csname PY@tok@c\endcsname{\let\PY@it=\textit\def\PY@tc##1{\textcolor[rgb]{0.25,0.50,0.50}{##1}}}
\expandafter\def\csname PY@tok@cp\endcsname{\def\PY@tc##1{\textcolor[rgb]{0.74,0.48,0.00}{##1}}}
\expandafter\def\csname PY@tok@k\endcsname{\let\PY@bf=\textbf\def\PY@tc##1{\textcolor[rgb]{0.00,0.50,0.00}{##1}}}
\expandafter\def\csname PY@tok@kp\endcsname{\def\PY@tc##1{\textcolor[rgb]{0.00,0.50,0.00}{##1}}}
\expandafter\def\csname PY@tok@kt\endcsname{\def\PY@tc##1{\textcolor[rgb]{0.69,0.00,0.25}{##1}}}
\expandafter\def\csname PY@tok@o\endcsname{\def\PY@tc##1{\textcolor[rgb]{0.40,0.40,0.40}{##1}}}
\expandafter\def\csname PY@tok@ow\endcsname{\let\PY@bf=\textbf\def\PY@tc##1{\textcolor[rgb]{0.67,0.13,1.00}{##1}}}
\expandafter\def\csname PY@tok@nb\endcsname{\def\PY@tc##1{\textcolor[rgb]{0.00,0.50,0.00}{##1}}}
\expandafter\def\csname PY@tok@nf\endcsname{\def\PY@tc##1{\textcolor[rgb]{0.00,0.00,1.00}{##1}}}
\expandafter\def\csname PY@tok@nc\endcsname{\let\PY@bf=\textbf\def\PY@tc##1{\textcolor[rgb]{0.00,0.00,1.00}{##1}}}
\expandafter\def\csname PY@tok@nn\endcsname{\let\PY@bf=\textbf\def\PY@tc##1{\textcolor[rgb]{0.00,0.00,1.00}{##1}}}
\expandafter\def\csname PY@tok@ne\endcsname{\let\PY@bf=\textbf\def\PY@tc##1{\textcolor[rgb]{0.82,0.25,0.23}{##1}}}
\expandafter\def\csname PY@tok@nv\endcsname{\def\PY@tc##1{\textcolor[rgb]{0.10,0.09,0.49}{##1}}}
\expandafter\def\csname PY@tok@no\endcsname{\def\PY@tc##1{\textcolor[rgb]{0.53,0.00,0.00}{##1}}}
\expandafter\def\csname PY@tok@nl\endcsname{\def\PY@tc##1{\textcolor[rgb]{0.63,0.63,0.00}{##1}}}
\expandafter\def\csname PY@tok@ni\endcsname{\let\PY@bf=\textbf\def\PY@tc##1{\textcolor[rgb]{0.60,0.60,0.60}{##1}}}
\expandafter\def\csname PY@tok@na\endcsname{\def\PY@tc##1{\textcolor[rgb]{0.49,0.56,0.16}{##1}}}
\expandafter\def\csname PY@tok@nt\endcsname{\let\PY@bf=\textbf\def\PY@tc##1{\textcolor[rgb]{0.00,0.50,0.00}{##1}}}
\expandafter\def\csname PY@tok@nd\endcsname{\def\PY@tc##1{\textcolor[rgb]{0.67,0.13,1.00}{##1}}}
\expandafter\def\csname PY@tok@s\endcsname{\def\PY@tc##1{\textcolor[rgb]{0.73,0.13,0.13}{##1}}}
\expandafter\def\csname PY@tok@sd\endcsname{\let\PY@it=\textit\def\PY@tc##1{\textcolor[rgb]{0.73,0.13,0.13}{##1}}}
\expandafter\def\csname PY@tok@si\endcsname{\let\PY@bf=\textbf\def\PY@tc##1{\textcolor[rgb]{0.73,0.40,0.53}{##1}}}
\expandafter\def\csname PY@tok@se\endcsname{\let\PY@bf=\textbf\def\PY@tc##1{\textcolor[rgb]{0.73,0.40,0.13}{##1}}}
\expandafter\def\csname PY@tok@sr\endcsname{\def\PY@tc##1{\textcolor[rgb]{0.73,0.40,0.53}{##1}}}
\expandafter\def\csname PY@tok@ss\endcsname{\def\PY@tc##1{\textcolor[rgb]{0.10,0.09,0.49}{##1}}}
\expandafter\def\csname PY@tok@sx\endcsname{\def\PY@tc##1{\textcolor[rgb]{0.00,0.50,0.00}{##1}}}
\expandafter\def\csname PY@tok@m\endcsname{\def\PY@tc##1{\textcolor[rgb]{0.40,0.40,0.40}{##1}}}
\expandafter\def\csname PY@tok@gh\endcsname{\let\PY@bf=\textbf\def\PY@tc##1{\textcolor[rgb]{0.00,0.00,0.50}{##1}}}
\expandafter\def\csname PY@tok@gu\endcsname{\let\PY@bf=\textbf\def\PY@tc##1{\textcolor[rgb]{0.50,0.00,0.50}{##1}}}
\expandafter\def\csname PY@tok@gd\endcsname{\def\PY@tc##1{\textcolor[rgb]{0.63,0.00,0.00}{##1}}}
\expandafter\def\csname PY@tok@gi\endcsname{\def\PY@tc##1{\textcolor[rgb]{0.00,0.63,0.00}{##1}}}
\expandafter\def\csname PY@tok@gr\endcsname{\def\PY@tc##1{\textcolor[rgb]{1.00,0.00,0.00}{##1}}}
\expandafter\def\csname PY@tok@ge\endcsname{\let\PY@it=\textit}
\expandafter\def\csname PY@tok@gs\endcsname{\let\PY@bf=\textbf}
\expandafter\def\csname PY@tok@gp\endcsname{\let\PY@bf=\textbf\def\PY@tc##1{\textcolor[rgb]{0.00,0.00,0.50}{##1}}}
\expandafter\def\csname PY@tok@go\endcsname{\def\PY@tc##1{\textcolor[rgb]{0.53,0.53,0.53}{##1}}}
\expandafter\def\csname PY@tok@gt\endcsname{\def\PY@tc##1{\textcolor[rgb]{0.00,0.27,0.87}{##1}}}
\expandafter\def\csname PY@tok@err\endcsname{\def\PY@bc##1{\setlength{\fboxsep}{0pt}\fcolorbox[rgb]{1.00,0.00,0.00}{1,1,1}{\strut ##1}}}
\expandafter\def\csname PY@tok@kc\endcsname{\let\PY@bf=\textbf\def\PY@tc##1{\textcolor[rgb]{0.00,0.50,0.00}{##1}}}
\expandafter\def\csname PY@tok@kd\endcsname{\let\PY@bf=\textbf\def\PY@tc##1{\textcolor[rgb]{0.00,0.50,0.00}{##1}}}
\expandafter\def\csname PY@tok@kn\endcsname{\let\PY@bf=\textbf\def\PY@tc##1{\textcolor[rgb]{0.00,0.50,0.00}{##1}}}
\expandafter\def\csname PY@tok@kr\endcsname{\let\PY@bf=\textbf\def\PY@tc##1{\textcolor[rgb]{0.00,0.50,0.00}{##1}}}
\expandafter\def\csname PY@tok@bp\endcsname{\def\PY@tc##1{\textcolor[rgb]{0.00,0.50,0.00}{##1}}}
\expandafter\def\csname PY@tok@fm\endcsname{\def\PY@tc##1{\textcolor[rgb]{0.00,0.00,1.00}{##1}}}
\expandafter\def\csname PY@tok@vc\endcsname{\def\PY@tc##1{\textcolor[rgb]{0.10,0.09,0.49}{##1}}}
\expandafter\def\csname PY@tok@vg\endcsname{\def\PY@tc##1{\textcolor[rgb]{0.10,0.09,0.49}{##1}}}
\expandafter\def\csname PY@tok@vi\endcsname{\def\PY@tc##1{\textcolor[rgb]{0.10,0.09,0.49}{##1}}}
\expandafter\def\csname PY@tok@vm\endcsname{\def\PY@tc##1{\textcolor[rgb]{0.10,0.09,0.49}{##1}}}
\expandafter\def\csname PY@tok@sa\endcsname{\def\PY@tc##1{\textcolor[rgb]{0.73,0.13,0.13}{##1}}}
\expandafter\def\csname PY@tok@sb\endcsname{\def\PY@tc##1{\textcolor[rgb]{0.73,0.13,0.13}{##1}}}
\expandafter\def\csname PY@tok@sc\endcsname{\def\PY@tc##1{\textcolor[rgb]{0.73,0.13,0.13}{##1}}}
\expandafter\def\csname PY@tok@dl\endcsname{\def\PY@tc##1{\textcolor[rgb]{0.73,0.13,0.13}{##1}}}
\expandafter\def\csname PY@tok@s2\endcsname{\def\PY@tc##1{\textcolor[rgb]{0.73,0.13,0.13}{##1}}}
\expandafter\def\csname PY@tok@sh\endcsname{\def\PY@tc##1{\textcolor[rgb]{0.73,0.13,0.13}{##1}}}
\expandafter\def\csname PY@tok@s1\endcsname{\def\PY@tc##1{\textcolor[rgb]{0.73,0.13,0.13}{##1}}}
\expandafter\def\csname PY@tok@mb\endcsname{\def\PY@tc##1{\textcolor[rgb]{0.40,0.40,0.40}{##1}}}
\expandafter\def\csname PY@tok@mf\endcsname{\def\PY@tc##1{\textcolor[rgb]{0.40,0.40,0.40}{##1}}}
\expandafter\def\csname PY@tok@mh\endcsname{\def\PY@tc##1{\textcolor[rgb]{0.40,0.40,0.40}{##1}}}
\expandafter\def\csname PY@tok@mi\endcsname{\def\PY@tc##1{\textcolor[rgb]{0.40,0.40,0.40}{##1}}}
\expandafter\def\csname PY@tok@il\endcsname{\def\PY@tc##1{\textcolor[rgb]{0.40,0.40,0.40}{##1}}}
\expandafter\def\csname PY@tok@mo\endcsname{\def\PY@tc##1{\textcolor[rgb]{0.40,0.40,0.40}{##1}}}
\expandafter\def\csname PY@tok@ch\endcsname{\let\PY@it=\textit\def\PY@tc##1{\textcolor[rgb]{0.25,0.50,0.50}{##1}}}
\expandafter\def\csname PY@tok@cm\endcsname{\let\PY@it=\textit\def\PY@tc##1{\textcolor[rgb]{0.25,0.50,0.50}{##1}}}
\expandafter\def\csname PY@tok@cpf\endcsname{\let\PY@it=\textit\def\PY@tc##1{\textcolor[rgb]{0.25,0.50,0.50}{##1}}}
\expandafter\def\csname PY@tok@c1\endcsname{\let\PY@it=\textit\def\PY@tc##1{\textcolor[rgb]{0.25,0.50,0.50}{##1}}}
\expandafter\def\csname PY@tok@cs\endcsname{\let\PY@it=\textit\def\PY@tc##1{\textcolor[rgb]{0.25,0.50,0.50}{##1}}}

\def\PYZbs{\char`\\}
\def\PYZus{\char`\_}
\def\PYZob{\char`\{}
\def\PYZcb{\char`\}}
\def\PYZca{\char`\^}
\def\PYZam{\char`\&}
\def\PYZlt{\char`\<}
\def\PYZgt{\char`\>}
\def\PYZsh{\char`\#}
\def\PYZpc{\char`\%}
\def\PYZdl{\char`\$}
\def\PYZhy{\char`\-}
\def\PYZsq{\char`\'}
\def\PYZdq{\char`\"}
\def\PYZti{\char`\~}
% for compatibility with earlier versions
\def\PYZat{@}
\def\PYZlb{[}
\def\PYZrb{]}
\makeatother


    % For linebreaks inside Verbatim environment from package fancyvrb. 
    \makeatletter
        \newbox\Wrappedcontinuationbox 
        \newbox\Wrappedvisiblespacebox 
        \newcommand*\Wrappedvisiblespace {\textcolor{red}{\textvisiblespace}} 
        \newcommand*\Wrappedcontinuationsymbol {\textcolor{red}{\llap{\tiny$\m@th\hookrightarrow$}}} 
        \newcommand*\Wrappedcontinuationindent {3ex } 
        \newcommand*\Wrappedafterbreak {\kern\Wrappedcontinuationindent\copy\Wrappedcontinuationbox} 
        % Take advantage of the already applied Pygments mark-up to insert 
        % potential linebreaks for TeX processing. 
        %        {, <, #, %, $, ' and ": go to next line. 
        %        _, }, ^, &, >, - and ~: stay at end of broken line. 
        % Use of \textquotesingle for straight quote. 
        \newcommand*\Wrappedbreaksatspecials {% 
            \def\PYGZus{\discretionary{\char`\_}{\Wrappedafterbreak}{\char`\_}}% 
            \def\PYGZob{\discretionary{}{\Wrappedafterbreak\char`\{}{\char`\{}}% 
            \def\PYGZcb{\discretionary{\char`\}}{\Wrappedafterbreak}{\char`\}}}% 
            \def\PYGZca{\discretionary{\char`\^}{\Wrappedafterbreak}{\char`\^}}% 
            \def\PYGZam{\discretionary{\char`\&}{\Wrappedafterbreak}{\char`\&}}% 
            \def\PYGZlt{\discretionary{}{\Wrappedafterbreak\char`\<}{\char`\<}}% 
            \def\PYGZgt{\discretionary{\char`\>}{\Wrappedafterbreak}{\char`\>}}% 
            \def\PYGZsh{\discretionary{}{\Wrappedafterbreak\char`\#}{\char`\#}}% 
            \def\PYGZpc{\discretionary{}{\Wrappedafterbreak\char`\%}{\char`\%}}% 
            \def\PYGZdl{\discretionary{}{\Wrappedafterbreak\char`\$}{\char`\$}}% 
            \def\PYGZhy{\discretionary{\char`\-}{\Wrappedafterbreak}{\char`\-}}% 
            \def\PYGZsq{\discretionary{}{\Wrappedafterbreak\textquotesingle}{\textquotesingle}}% 
            \def\PYGZdq{\discretionary{}{\Wrappedafterbreak\char`\"}{\char`\"}}% 
            \def\PYGZti{\discretionary{\char`\~}{\Wrappedafterbreak}{\char`\~}}% 
        } 
        % Some characters . , ; ? ! / are not pygmentized. 
        % This macro makes them "active" and they will insert potential linebreaks 
        \newcommand*\Wrappedbreaksatpunct {% 
            \lccode`\~`\.\lowercase{\def~}{\discretionary{\hbox{\char`\.}}{\Wrappedafterbreak}{\hbox{\char`\.}}}% 
            \lccode`\~`\,\lowercase{\def~}{\discretionary{\hbox{\char`\,}}{\Wrappedafterbreak}{\hbox{\char`\,}}}% 
            \lccode`\~`\;\lowercase{\def~}{\discretionary{\hbox{\char`\;}}{\Wrappedafterbreak}{\hbox{\char`\;}}}% 
            \lccode`\~`\:\lowercase{\def~}{\discretionary{\hbox{\char`\:}}{\Wrappedafterbreak}{\hbox{\char`\:}}}% 
            \lccode`\~`\?\lowercase{\def~}{\discretionary{\hbox{\char`\?}}{\Wrappedafterbreak}{\hbox{\char`\?}}}% 
            \lccode`\~`\!\lowercase{\def~}{\discretionary{\hbox{\char`\!}}{\Wrappedafterbreak}{\hbox{\char`\!}}}% 
            \lccode`\~`\/\lowercase{\def~}{\discretionary{\hbox{\char`\/}}{\Wrappedafterbreak}{\hbox{\char`\/}}}% 
            \catcode`\.\active
            \catcode`\,\active 
            \catcode`\;\active
            \catcode`\:\active
            \catcode`\?\active
            \catcode`\!\active
            \catcode`\/\active 
            \lccode`\~`\~ 	
        }
    \makeatother

    \let\OriginalVerbatim=\Verbatim
    \makeatletter
    \renewcommand{\Verbatim}[1][1]{%
        %\parskip\z@skip
        \sbox\Wrappedcontinuationbox {\Wrappedcontinuationsymbol}%
        \sbox\Wrappedvisiblespacebox {\FV@SetupFont\Wrappedvisiblespace}%
        \def\FancyVerbFormatLine ##1{\hsize\linewidth
            \vtop{\raggedright\hyphenpenalty\z@\exhyphenpenalty\z@
                \doublehyphendemerits\z@\finalhyphendemerits\z@
                \strut ##1\strut}%
        }%
        % If the linebreak is at a space, the latter will be displayed as visible
        % space at end of first line, and a continuation symbol starts next line.
        % Stretch/shrink are however usually zero for typewriter font.
        \def\FV@Space {%
            \nobreak\hskip\z@ plus\fontdimen3\font minus\fontdimen4\font
            \discretionary{\copy\Wrappedvisiblespacebox}{\Wrappedafterbreak}
            {\kern\fontdimen2\font}%
        }%
        
        % Allow breaks at special characters using \PYG... macros.
        \Wrappedbreaksatspecials
        % Breaks at punctuation characters . , ; ? ! and / need catcode=\active 	
        \OriginalVerbatim[#1,codes*=\Wrappedbreaksatpunct]%
    }
    \makeatother

    % Exact colors from NB
    \definecolor{incolor}{HTML}{303F9F}
    \definecolor{outcolor}{HTML}{D84315}
    \definecolor{cellborder}{HTML}{CFCFCF}
    \definecolor{cellbackground}{HTML}{F7F7F7}
    
    % prompt
    \makeatletter
    \newcommand{\boxspacing}{\kern\kvtcb@left@rule\kern\kvtcb@boxsep}
    \makeatother
    \newcommand{\prompt}[4]{
        \ttfamily\llap{{\color{#2}[#3]:\hspace{3pt}#4}}\vspace{-\baselineskip}
    }
    

    
    % Prevent overflowing lines due to hard-to-break entities
    \sloppy 
    % Setup hyperref package
    \hypersetup{
      breaklinks=true,  % so long urls are correctly broken across lines
      colorlinks=true,
      urlcolor=urlcolor,
      linkcolor=linkcolor,
      citecolor=citecolor,
      }
    % Slightly bigger margins than the latex defaults
    
    \geometry{verbose,tmargin=1in,bmargin=1in,lmargin=1in,rmargin=1in}
    
    

\begin{document}
    
    \maketitle
    
    

    
    Universidad Tecnológica Nacional. Facultad Regional Mendoza. Teoría de
los Circuitos 1

\hypertarget{resoluciuxf3n-de-circuitos-simples}{%
\section{Resolución de circuitos
simples:}\label{resoluciuxf3n-de-circuitos-simples}}

    \textbf{Ejercicio 1:} Dado el circuito de la figura, encontrar el valor
de la intensidad de corriente \(I\) y la caída de tensión en \(R_2\).
Demostrar que se cumple la Ley de Kirchhoff para las tensiones.

    \begin{center}
    \adjustimage{max size={0.9\linewidth}{0.9\paperheight}}{Trabajo práctico Nº 1_files/Trabajo práctico Nº 1_2_0.png}
    \end{center}
    { \hspace*{\fill} \\}
    
    Para resolver el circuito necesitamos encontrar la tensión y la
corriente en cada elemento. Podemos identificar 9 incógnitas:
\(I_{R1}\), \(I_{R2}\), \(I_{R3}\), \(I_{V1}\), \(I_{V2}\), \(I_{V3}\),
\(V_{R1}\), \(V_{R2}\), \(V_{R3}\), ya que la tensión de las fuentes es
conocida.

\textbf{Dada Ley de Kirchhoff para la corriente: La suma algebraica de
todas las corrientes en cualquier nodo de un circuito es igual a cero},
podemos deducir que la intensidad de corriente en todos los elementos
tiene la misma magnitud y sentido, ya que la corriente que sale de un
elemento ingresa a un nodo donde el único camino posible para la
corriente es el siguiente elemento.

Por ejemplo, la corriente \(I_{v1}\) que circula por el generador
\(V_1\) ingresa al nodo \textbf{1} y la corriente \(I_{R1}\) sale del
nodo \textbf{1}, por ley de Kirchhoff de las corrientes
\(I_{v1}+I_{R1}=0\) se deduce que \(I_{v1}=I_{R1}\).

Siguiendo la misma lógica en todos los nodos obtenemos que
\[I_{V1}=I_{R1}=I_{V2}=I_{R2}=I_{V3}=I_{R3}=I\]

\textbf{Planteando la Ley Kirchhoff para las tensiones: La suma
algebraica de todas las tensiones a los largo de cualquier camino
cerrado en un circuito es igual a cero.}

\(V_1-V_{R1}+V_2-V_{R2}-V_3-V_{R3}=0\)

\(V_1+V_2-V_3=V_{R1}+V_{R2}+V_{R3}\)

\(5V+3V-10V=-2=V_{R1}+V_{R2}+V_{R3}\)

Y aplicando la Ley de Ohm en \(V_{R1}\), \(V_{R2}\) y \(V_{R3}\):

\(-2V=I\times R_1+I \times R_2+I \times R_3=I(R_1+R_2+R_3)=I(2\Omega+4\Omega+2\Omega)\)

Por lo que \(I=\frac{-2V}{8\Omega}=250mA\)

    Habiendo calculado la corriente que circula por todos los elementos,
podemos ahora calcular la tensión en cada resistencia. Simplemente
aplicamos la ley de Ohm. Por ejemplo:
\(V_{R1}=I_1\times R=-\frac{1}{4}A\times 2\Omega=-\frac{1}{2}V\)

    \textbf{Ejercicio 2}: El circuito de la siguiente figura es conocido
como divisor de tensión ya que reparte la tensión de una fuente entre
dos o más resistencias conectadas en serie. Obtenga la expresión de la
tensión en cada una de las resistencias del circuito.

    \begin{center}
    \adjustimage{max size={0.9\linewidth}{0.9\paperheight}}{Trabajo práctico Nº 1_files/Trabajo práctico Nº 1_6_0.png}
    \end{center}
    { \hspace*{\fill} \\}
    
    Rta: \(V_{R1}=\frac{V_1 \times R_{1}}{R_1+R_2}\),
\(V_{R2}=\frac{V_1 \times R_{2}}{R_1+R_2}\)

\textbf{Eejercicio 3:} Plantear la expresión de la diferencia de
potencial en función de la corriente entre los terminales \(a_x\) y
\(b_x\) en cada una de las ramas presentadas a continuación. Suponer que
los terminale \(a_x\) y \(b_x\) están conectados a un circuito (no
graficado) de manera que existen caminos para que circulen las
corrientes allí definidas.

\begin{enumerate}
\def\labelenumi{\alph{enumi})}
\tightlist
\item
\end{enumerate}

    \begin{center}
    \adjustimage{max size={0.9\linewidth}{0.9\paperheight}}{Trabajo práctico Nº 1_files/Trabajo práctico Nº 1_8_0.png}
    \end{center}
    { \hspace*{\fill} \\}
    
    Por ejemplo: \(V_{a1-b1}=V-V_{R_1}=V-I_1 \times R_1\)

    \begin{enumerate}
\def\labelenumi{\alph{enumi})}
\setcounter{enumi}{1}
\tightlist
\item
\end{enumerate}

    \begin{center}
    \adjustimage{max size={0.9\linewidth}{0.9\paperheight}}{Trabajo práctico Nº 1_files/Trabajo práctico Nº 1_11_0.png}
    \end{center}
    { \hspace*{\fill} \\}
    
    Notar en este caso que el sentido de la corriente a través de las
resistencias es contrario a la convención de signos, por ende cuando
reemplazamos un término planteando la ley de ohm, deberá escribirse con
el signo cambiado.

Por ejemplo: \(V_{a1-b1}=V-V_{R_1}\) AP, aplicando ley de Ohm
\(V_{a1-b1}=V-(-)I_1 \times R_1=V+I_1 \times R_1\)

    \textbf{Ejercicio 5:} Calcular el voltaje de las fuentes de tensión de
los siguientes circuitos.

\begin{enumerate}
\def\labelenumi{\alph{enumi})}
\tightlist
\item
\end{enumerate}

    \begin{center}
    \adjustimage{max size={0.9\linewidth}{0.9\paperheight}}{Trabajo práctico Nº 1_files/Trabajo práctico Nº 1_14_0.png}
    \end{center}
    { \hspace*{\fill} \\}
    
    Rta: \(V_1=12V\)

\begin{enumerate}
\def\labelenumi{\alph{enumi})}
\setcounter{enumi}{1}
\tightlist
\item
\end{enumerate}

    \begin{center}
    \adjustimage{max size={0.9\linewidth}{0.9\paperheight}}{Trabajo práctico Nº 1_files/Trabajo práctico Nº 1_16_0.png}
    \end{center}
    { \hspace*{\fill} \\}
    
    Rta: \(V_1=50V\)

\textbf{Ejercicio 6}: Dado el circuito de la figura, calcular el valor
de la tensión \(V\) y el de la corriente que circula por \(R_1\),
\(R_2\), \(R_3\). Demostrar que se cumple la Ley de Kirchhoff para las
corrientes.

    \begin{center}
    \adjustimage{max size={0.9\linewidth}{0.9\paperheight}}{Trabajo práctico Nº 1_files/Trabajo práctico Nº 1_18_0.png}
    \end{center}
    { \hspace*{\fill} \\}
    
    El circuito solo posee dos nodos por lo que puede ser redibujado de la
siguiente manera:

    \begin{center}
    \adjustimage{max size={0.9\linewidth}{0.9\paperheight}}{Trabajo práctico Nº 1_files/Trabajo práctico Nº 1_20_0.png}
    \end{center}
    { \hspace*{\fill} \\}
    
    Se dibujó de esta manera sólo con fines didácticos, aunque se obtienen
exáctamente las mismas concluciones analizando el circuito original.

Note que todos los elementos siguen conectados entre los mismos nodos,
por lo que los circuitos son equivalentes. Vemos que al nodo \textbf{1}
ingresa \(I_1\) y egresa \(I_2\), por lo que la corriente neta debido a
las fuentes de corriente que ingresa al nodo \textbf{1} es
\(I_1-I_2=5A\)

    \begin{center}
    \adjustimage{max size={0.9\linewidth}{0.9\paperheight}}{Trabajo práctico Nº 1_files/Trabajo práctico Nº 1_22_0.png}
    \end{center}
    { \hspace*{\fill} \\}
    
    La resistencia equivalente entre los nodos \textbf{1} y \textbf{0} es
\(R_{eq}= \left( \frac{1}{R_1}+\frac{1}{R_2}+\frac{1}{R_3} \right) ^{-1}\)

    Redibujando el circuito queda:

    \begin{center}
    \adjustimage{max size={0.9\linewidth}{0.9\paperheight}}{Trabajo práctico Nº 1_files/Trabajo práctico Nº 1_25_0.png}
    \end{center}
    { \hspace*{\fill} \\}
    
    La tensión entre el nodo \textbf{1} y \textbf{2} es
\(V=I \times R_{eq} = 5A \times 0.8 \Omega = 4V\)

    Sabiendo la diferencia de tensión entre los nodos, podemos calcular la
corriente en cada resistencia por Ley de Ohm.

\(I_{R1} = \frac{V}{R_{1}}=\frac{4V}{2\Omega}=2A\),\\
\(I_{R2} = \frac{V}{R_{2}}=\frac{4V}{4\Omega}=1A\)\\
\(I_{R3} = \frac{V}{R_{3}}=\frac{4V}{2\Omega}=2A\)

Comprobamos que se cumple la ley de las corrientes de Kirchhoff en el
nodo \textbf{1}:

\(I_1-I_{R_1}-I_{R_2}-I_{R_3}-I_2=0\)

\(10A-2A-1A-2A-5A=0\)

    \textbf{Ejercicio 7:} El circuito de la siguiente figura es conocido
como divisor de corriente ya que reparte la corriente de una fuente
entre dos o más resistencias conectadas en paralelo. Obtenga la
expresión de la corriente en cada una de las resistencias del circuito.

    \begin{center}
    \adjustimage{max size={0.9\linewidth}{0.9\paperheight}}{Trabajo práctico Nº 1_files/Trabajo práctico Nº 1_29_0.png}
    \end{center}
    { \hspace*{\fill} \\}
    
    \(I_{R1}=\frac{I \times R_{2}}{R_1+R_2}\),
\(I_{R2}=\frac{I \times R_{1}}{R_1+R_2}\)

\textbf{Ejercicio 8:} Investigue el principio de funcionamiento de un
voltímetro y de un amperímetro. Explique cómo se deben conectar en un
circuito y describa cómo es desaeble que sea la resistencia interna de
cada uno de estos instrumentos y justifique

    \textbf{Ejercicio 9:} En una toma de alimentación de una instalación
domiciliaria (enchufe) encontramos una tensión eficaz de \(220 V\). A
estas tomas se conectan todos los elctrodomésticos y equipos
electrónicos/electromecánicos que hay en un hogar. Si todos estos
ertefactos son alimentados por la misma tensión (\(220V\)):

\begin{enumerate}
\def\labelenumi{\alph{enumi})}
\item
  Deduzca cómo están interconectados a la red eléctrica (serio o
  paralelo) todos los artefactos de un hogar.
\item
  Dibuje un diagrama básico de tres artefactos conectados en un hogar.
  (Representar el tomacorrientes como una fuente de tensión y los
  diferentes artefactos como resistencias)
\end{enumerate}

    \textbf{Ejercicio 10:} En el siguiente diagrama se presenta un circuito
eléctrico que se utiliza para medir resistencias conocido como
\textbf{puente de Wheatstone}. Invesitigue sobre este circuito y
describa su funcionamiento

Entre \textbf{a} y \textbf{b} se conecta un voltímetro. \(R_{1}\) y
\(R_{3}\) son resistencias conocidas y \(R_{2}\) una resistencia
variable. \(R_{x}\) es la resistencia a medir.

    \begin{center}
    \adjustimage{max size={0.9\linewidth}{0.9\paperheight}}{Trabajo práctico Nº 1_files/Trabajo práctico Nº 1_33_0.png}
    \end{center}
    { \hspace*{\fill} \\}
    
    \begin{center}\rule{0.5\linewidth}{0.5pt}\end{center}

\hypertarget{principios-fundamentales.}{%
\section{Principios fundamentales.}\label{principios-fundamentales.}}

\begin{enumerate}
\def\labelenumi{\arabic{enumi}.}
\tightlist
\item
  Teoremas de Thévenin y Norton
\item
  Teorema de la sustitución.
\item
  Principios de dualidad, linealidad y superposición.
\item
  Teorema de reducción de generadores de Millman.
\end{enumerate}

    \textbf{Ejercicio 11:} Obtener el equivalente de Thévenin de los
siguientes circuitos:

\begin{enumerate}
\def\labelenumi{\alph{enumi})}
\tightlist
\item
\end{enumerate}

    \begin{center}
    \adjustimage{max size={0.9\linewidth}{0.9\paperheight}}{Trabajo práctico Nº 1_files/Trabajo práctico Nº 1_36_0.png}
    \end{center}
    { \hspace*{\fill} \\}
    
    Rta: \(V_{Th}=I \times R_{1}\) y \(R_{Th}= R_1\)

    \begin{enumerate}
\def\labelenumi{\alph{enumi})}
\setcounter{enumi}{1}
\tightlist
\item
\end{enumerate}

    \begin{center}
    \adjustimage{max size={0.9\linewidth}{0.9\paperheight}}{Trabajo práctico Nº 1_files/Trabajo práctico Nº 1_39_0.png}
    \end{center}
    { \hspace*{\fill} \\}
    
    Rta: \(V_{Th}=-10V\) y \(R_{Th}=2\Omega\)

\begin{enumerate}
\def\labelenumi{\alph{enumi})}
\setcounter{enumi}{2}
\tightlist
\item
\end{enumerate}

    \begin{center}
    \adjustimage{max size={0.9\linewidth}{0.9\paperheight}}{Trabajo práctico Nº 1_files/Trabajo práctico Nº 1_41_0.png}
    \end{center}
    { \hspace*{\fill} \\}
    
    Rta: \(V_{Th}=-32V\) y \(R_{Th}=4\Omega\)

\textbf{Ejercicio 12:} Obtener el equivalente de Norton de los
siguientes circuitos:

\begin{enumerate}
\def\labelenumi{\alph{enumi})}
\tightlist
\item
\end{enumerate}

    \begin{center}
    \adjustimage{max size={0.9\linewidth}{0.9\paperheight}}{Trabajo práctico Nº 1_files/Trabajo práctico Nº 1_43_0.png}
    \end{center}
    { \hspace*{\fill} \\}
    
    Rta: \(I_{N}=2,5A\) y \(R_{N}=4\Omega\)

\begin{enumerate}
\def\labelenumi{\alph{enumi})}
\setcounter{enumi}{1}
\tightlist
\item
\end{enumerate}

    \begin{center}
    \adjustimage{max size={0.9\linewidth}{0.9\paperheight}}{Trabajo práctico Nº 1_files/Trabajo práctico Nº 1_45_0.png}
    \end{center}
    { \hspace*{\fill} \\}
    
    Rta: \(I_{N}=4A\) y \(R_{N}=5\Omega\)

\textbf{Ejercicio 13:} Dado los siguientes circuitos, encontrar el
equivalente de Norton y de Thévenin aplicando el principio de
superposición.

\begin{enumerate}
\def\labelenumi{\alph{enumi})}
\tightlist
\item
\end{enumerate}

    \begin{center}
    \adjustimage{max size={0.9\linewidth}{0.9\paperheight}}{Trabajo práctico Nº 1_files/Trabajo práctico Nº 1_47_0.png}
    \end{center}
    { \hspace*{\fill} \\}
    
    Rta: \(V_{Th}=288V\) y \(R_{Th}=2,97\Omega\)

\(I_{N}=97A\) y \(R_{N}=2,97\Omega\)

\begin{enumerate}
\def\labelenumi{\alph{enumi})}
\setcounter{enumi}{1}
\tightlist
\item
\end{enumerate}

    \begin{center}
    \adjustimage{max size={0.9\linewidth}{0.9\paperheight}}{Trabajo práctico Nº 1_files/Trabajo práctico Nº 1_49_0.png}
    \end{center}
    { \hspace*{\fill} \\}
    
    Rta: \(V_{Th}=112V\) y \(R_{Th}=16\Omega\)

\(I_{N}=7A\) y \(R_{N}=16\Omega\)

\begin{enumerate}
\def\labelenumi{\alph{enumi})}
\setcounter{enumi}{2}
\tightlist
\item
\end{enumerate}

    \begin{center}
    \adjustimage{max size={0.9\linewidth}{0.9\paperheight}}{Trabajo práctico Nº 1_files/Trabajo práctico Nº 1_51_0.png}
    \end{center}
    { \hspace*{\fill} \\}
    
    Rta: \(V_{Th}=-7,5V\) y \(R_{Th}=7,5k\Omega\)

\(I_{N}=-1mA\) y \(R_{N}=7,5k\Omega\)

\textbf{Ejercicio 14:} Encontrar el valor de \(I_{0}\) aplicando
transformación de fuentes.

    \begin{center}
    \adjustimage{max size={0.9\linewidth}{0.9\paperheight}}{Trabajo práctico Nº 1_files/Trabajo práctico Nº 1_53_0.png}
    \end{center}
    { \hspace*{\fill} \\}
    
    Rta: \(I_0=4,31mA\)

    \begin{center}\rule{0.5\linewidth}{0.5pt}\end{center}

\hypertarget{resoluciuxf3n-sistemuxe1tica-de-circuitos.}{%
\section{Resolución sistemática de
circuitos.}\label{resoluciuxf3n-sistemuxe1tica-de-circuitos.}}

\begin{enumerate}
\def\labelenumi{\arabic{enumi}.}
\tightlist
\item
  Método de análisis de las corrientes en las mallas (método de
  Maxwell).
\item
  Método de análisis de las tensiones nodales.
\end{enumerate}

\begin{center}\rule{0.5\linewidth}{0.5pt}\end{center}

    \textbf{Ejercicio 15: Corrientes de mallas}

Utilizando el método de las corrientes de mallas calcular las incógnitas
solicitadas en los siguientes circuitos realizando el siguiente
procedimiento:

\begin{enumerate}
\def\labelenumi{\alph{enumi})}
\item
  Graficar el diagrama topológico del circuito.
\item
  Establecer el sentido de las corrientes de malla en una sola dirección
  para todas las mallas, por ejemplo en sentido horario.
\item
  Escribir las ecuaciones de Kirchhoff de tensión para cada malla
  topológica, en función de las corrientes de mallas planteadas.
\item
  Resuelva el sistema de ecuaciones resultantes.
\item
  Calcule el valor de la incógnita solicitada en el diagrama
\end{enumerate}

    15.1)

    \begin{center}
    \adjustimage{max size={0.9\linewidth}{0.9\paperheight}}{Trabajo práctico Nº 1_files/Trabajo práctico Nº 1_58_0.png}
    \end{center}
    { \hspace*{\fill} \\}
    
    Rta: \(I_x=1A\)

15.2)

    \begin{center}
    \adjustimage{max size={0.9\linewidth}{0.9\paperheight}}{Trabajo práctico Nº 1_files/Trabajo práctico Nº 1_60_0.png}
    \end{center}
    { \hspace*{\fill} \\}
    
    Rta: \(V_x=\frac{-7}{5} V\)

15.3)

    \begin{center}
    \adjustimage{max size={0.9\linewidth}{0.9\paperheight}}{Trabajo práctico Nº 1_files/Trabajo práctico Nº 1_62_0.png}
    \end{center}
    { \hspace*{\fill} \\}
    
    Rta: \(I_x=-1,46A\)

\textbf{Ejercicio 16: Corrientes de mallas con ramas virtuales}

Calcule las incógnitas solicitadas en los siguientes circuitos aplicando
el método de corrientes de malla:

16.1)

    \begin{center}
    \adjustimage{max size={0.9\linewidth}{0.9\paperheight}}{Trabajo práctico Nº 1_files/Trabajo práctico Nº 1_64_0.png}
    \end{center}
    { \hspace*{\fill} \\}
    
    Rta: \(V_x=-\frac{4}{3}V\)

16.2)

    \begin{center}
    \adjustimage{max size={0.9\linewidth}{0.9\paperheight}}{Trabajo práctico Nº 1_files/Trabajo práctico Nº 1_66_0.png}
    \end{center}
    { \hspace*{\fill} \\}
    
    Rta: \(V_x=\frac{5}{2}V\)

16.3)

    \begin{center}
    \adjustimage{max size={0.9\linewidth}{0.9\paperheight}}{Trabajo práctico Nº 1_files/Trabajo práctico Nº 1_68_0.png}
    \end{center}
    { \hspace*{\fill} \\}
    
    Rta: \(V_x=-1,91V\)

\textbf{Ejercicio 17: Tensiones nodales}

Utilizando el método de las tensiones nodales calcular las incógnitas
solicitadas en los siguientes circuitos realizando el siguiente
procedimiento:

\begin{enumerate}
\def\labelenumi{\alph{enumi})}
\item
  Graficar el diagrama topológico
\item
  Definir un nodo de referencia y denominar cada uno de los nodos con
  una letra.
\item
  Escribir las ecuaciones de corrientes de Kirchhoff en función de las
  tensiones nodales para cada nodo topológico excepto para el nodo de
  referencia.
\item
  Resolver el sistema de ecuaciones obtenido.
\item
  Calcule el valor de la incógnita solicitada en el diagrama
\end{enumerate}

17.1)

    \begin{center}
    \adjustimage{max size={0.9\linewidth}{0.9\paperheight}}{Trabajo práctico Nº 1_files/Trabajo práctico Nº 1_70_0.png}
    \end{center}
    { \hspace*{\fill} \\}
    
    Rta: \(V_x=5,82V\)

17.2)

    \begin{center}
    \adjustimage{max size={0.9\linewidth}{0.9\paperheight}}{Trabajo práctico Nº 1_files/Trabajo práctico Nº 1_72_0.png}
    \end{center}
    { \hspace*{\fill} \\}
    
    Rta: \(I_x=-\frac{2}{13}A\)

17.3)

    \begin{center}
    \adjustimage{max size={0.9\linewidth}{0.9\paperheight}}{Trabajo práctico Nº 1_files/Trabajo práctico Nº 1_74_0.png}
    \end{center}
    { \hspace*{\fill} \\}
    
    Rta: \(I_x=4,75A\)

\textbf{Ejercicio 18: Tensiones nodales con super-nodos}

Resuelva los siguientes ejercicios aplicando el método de tensiones
nodales.

\begin{enumerate}
\def\labelenumi{\arabic{enumi})}
\tightlist
\item
\end{enumerate}

    \begin{center}
    \adjustimage{max size={0.9\linewidth}{0.9\paperheight}}{Trabajo práctico Nº 1_files/Trabajo práctico Nº 1_76_0.png}
    \end{center}
    { \hspace*{\fill} \\}
    
    Rta: \(I_x=-1A\)

18.2)

    \begin{center}
    \adjustimage{max size={0.9\linewidth}{0.9\paperheight}}{Trabajo práctico Nº 1_files/Trabajo práctico Nº 1_78_0.png}
    \end{center}
    { \hspace*{\fill} \\}
    
    Rta: \(V_x=-5,56V\)

18.3)

    \begin{center}
    \adjustimage{max size={0.9\linewidth}{0.9\paperheight}}{Trabajo práctico Nº 1_files/Trabajo práctico Nº 1_80_0.png}
    \end{center}
    { \hspace*{\fill} \\}
    
    Rta: \(I_x=20A\)


    % Add a bibliography block to the postdoc
    
    
    
\end{document}
