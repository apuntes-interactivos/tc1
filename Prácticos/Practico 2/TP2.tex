% ***********************************************
%***********COSAS A MEJORAR O CORREGIR***********
%
%
%
%
%
%
%
% ***********************************************
% Preamble
% ---
\documentclass[10pt,a4paper]{article}

% Packages
% ---
\usepackage{float}
\usepackage[utf8]{inputenc}
\usepackage{amsmath}
\usepackage{mathtools}
\usepackage{amsfonts}
\usepackage{amssymb}
\usepackage{layout}
\usepackage{graphics}
\usepackage{graphicx}
\usepackage[spanish]{babel}
\usepackage{lastpage}
\usepackage{tabto}
\usepackage{xcolor}
\usepackage[a4paper, top=1 in,bottom=1in, left=1 in, right=0.7 in]{geometry}
\usepackage{fancyhdr}
\pagestyle{fancy}
\usepackage{comment}





\lhead{\footnotesize Teoría de los circuitos 1 - Año \the\year{} Ver. 2022.01.12}
\chead{}
\rhead{\footnotesize Trabajo Práctico N$^\text o$2.}
\lfoot{}
\cfoot{}
\rfoot{\footnotesize P\'agina \thepage\ de \pageref{LastPage}}
\renewcommand{\headrulewidth}{0.4pt}
\renewcommand{\footrulewidth}{0.4pt}

\title{
	\textsc{\includegraphics[width=0.35\textwidth]{logoUTN.jpg}} ~\\
	{\large Departamento de Electr\'onica}\\ 
	[0.1cm]
	{\Huge{Teoría de circuitos 1}} \\
	[0.25cm]
	{\Large{Trabajo Práctico N$^{\text {o}}$2}		\\
}}
\author{}
\date{}
    
\begin{document}
	\maketitle
	
\section{Régimen permanente}
\begin{enumerate}
	
% ***********************************************
% 					Ejercicio 1
% ***********************************************

\item Dado el siguiente circuito:

\begin{figure}[H]
	\centering
	\includegraphics[scale=0.7]{img/1.png}
	\label{e1}
\end{figure}

\begin{enumerate}

	 \item Calcular las reactancias. 

	 \item Calcular la impedancia total y expresarla en forma polar y binómica.

	 \item Calcular la corriente en forma polar y luego expresarla en función del tiempo.

	 \item Calcular la tensión en cada elemento en forma polar y luego expresarla en función del tiempo.

	 \item Graficar las tensiones de cada elemento en función del tiempo.

	 \item Graficar el diagrama fasorial.
	 
\end{enumerate}

Rtas: a) $X_C=10 \Omega$; $X_L=20 \Omega$; b)$Z_T=(10+j10)\Omega=\sqrt{200}\angle{45^{\circ}}\Omega$; c) $\bar{I}=2,12\angle{-45^{\circ}}[A]$; $i(t)=2,12\,sin(2.\pi 50Hz.t+\frac{\pi}{4})[A]$; d)$\bar{V}_R=21,2\angle{45^{\circ}}[V]$; $\bar{V}_C=21,2\angle{-45^{\circ}}[V]$; $\bar{V_L}=42,4\angle{135^{\circ}}[V]$\\

% ***********************************************
% 					Ejercicio 2
% ***********************************************


\item A una impedancia $Z= (10 + j15)\Omega$ se le aplica una tensión de $v(t) = 141,2\, cos(2000 t + \pi / 4) [V]$.

\begin{enumerate}
	\item Dibujar el diagrama del circuito.
	\item Calcular la corriente que circula en el circuito expresada en forma polar.
	\item Expresar la corriente en función del tiempo.
\end{enumerate}


Rta: b) $\bar{I}=7,83\angle{-11,31^{\circ}}[A]$, c) $i(t)=7,83\,cos(2000t-11,31^{\circ})[A]$
% ***********************************************
% 					Ejercicio 3
% ***********************************************

\item La corriente que circula en un circuito serie es $\bar{I}=(4 + j12) [A]$ cuando la tensión aplicada es de $V=180 \angle{55^{\circ}}[V]$

\begin{enumerate}
	
	\item Determinar la impedancia del circuito y expresarla en las formas polar y binómica.

	\item Indicar si la reactancia es capacitiva o inductiva. Justifique.

\end{enumerate}
	
Rta: a) $\bar{Z}=(13,64-j4,06)\Omega=14,23\angle{-16,57^{\circ}}\Omega$


% ***********************************************
% 					Ejercicio 4
% ***********************************************
\item La fuente de tensión  del circuito de la figura reproduce la función $v(t)=100\,cos(5 \times 10^3.t + 45^{\circ})$[V] 

\begin{figure}[H]
	\centering
	\includegraphics[scale=0.8]{img/2.png}
	\label{e2}
\end{figure}

hallar:

\begin{enumerate}

\item Intensidad de corriente total expresada en forma trigonométrica.

\item El valor de las corrientes $\bar{I}_1$ e $\bar{I}_2$ y las caídas de tensión en cada elemento expresando las respuestas en forma binómica.

\item Dos elementos en serie que produzcan la misma corriente $\bar{I_t}$.

\item Dos elementos en paralelo que produzcan la misma corriente $\bar{I_t}$. (ayuda: resolver pensando el circuito en términos de admitancias)

\end{enumerate}

Rta: a) $\bar{I}_t=(6,84+j17,21)[A]$ b) $\bar{I}_1=(9,67-j2,59)[A]$; $\bar{I}_2=(-2,82-j19,8)[A]$; c) $R=4,96\Omega$ $C=93,55\mu F$ d) $R=5.88\Omega$ $C=14.67\mu F$

% ***********************************************
% 					Ejercicio 5
% ***********************************************
\item Dado el siguiente circuito y siendo $\bar{V_1}=35,36 \angle 0^{\circ}[V]$ y $\bar{V_2}=35,36 \angle 90^{\circ}[V]$ , calcular:

\begin{enumerate}
	\item La impedancia equivalente del circuito.
	\item La corriente 
	\item La diferencia de potencial en cada elemento
	\item Graficar el diagrama fasorial que muestre la corriente y las tensiones en el circuito.
\end{enumerate}

\begin{figure}[H]
	\centering
	\includegraphics[scale=0.8]{img/3.png}
	\label{e5}
\end{figure}

Rta: a) $\bar{Z}_{eq}=(10-j10)\Omega$; b) $\bar{I}=3,536j[A]$; c) $\bar{V}_R=35,36j[V]$; $\bar{V}_L=-70,72[V]$; $\bar{V}_C=106,8[V]$

% ***********************************************
% 					Ejercicio 6
% ***********************************************
\item Un capacitor de $100 \mu F$ está conectado en serie con una bobina real que posee asociada una resistencia de $5 \Omega$ y una inductancia de $0,12 Hy$ . El módulo de la corriente que circula en el circuito es  $|\bar{I}|=64,85 [A]$. La corriente está retrasada $49,6^{\circ}$ con respecto a la tensión suministrada. La frecuencia de trabajo es de $50 Hz$ 

\begin{figure}[H]
	\centering
	\includegraphics[scale=0.8]{img/4.png}
	\label{e6}
\end{figure}

\begin{enumerate}
	\item Calcular el valor r.m.s. de la tensión en la bobina y en el capacitor 
	\item ¿Cuál es el valor de la tensión de alimentación?
	\item Indicar si el circuito se comporta de manera inductiva o capacitiva.
\end{enumerate}

Rta: $\bar{V}_L=2466,25 \angle{32,85°}[V]$; $\bar{V}_C=2064,18\angle{-139,6°}[V]$; $\bar{V}_G=500\angle{0°}[V]$
% ***********************************************
% 					Ejercicio  7
% ***********************************************
\item Dado el siguiente circuito calcular:

\begin{enumerate}
	\item La impedancia del circuito.
	\item La corriente total
	\item La corriente que circula en cada elemento.
	\item La tensión entre los nodos A-B y B-C.
\end{enumerate}


\begin{figure}[H]
	\centering
	\includegraphics[scale=0.9]{img/5.png}
	\label{e7}
\end{figure}

Rta: a) $Z=18,97\angle{-18,43^{\circ}}\Omega$; b) $\bar{I}_{total}=(1+j0,33)[A]$; c) $\bar{I}_1=(0,67+j0,67)[A]$; $\bar{I}_2=(0,33-j0,33)[A]$  ;$\bar{I}_3=(6,66-j3,33)[A]$; $\bar{I}_4=(3,33+j6,66)[A]$ d) $\bar{V}_{AB}=(6,67+j6,67)[V]$; $\bar{V}_{BC}=(13,33-j6,67)[V]$
% ***********************************************
% 					Ejercicio 8
% ***********************************************
\item Dado el siguiente circuito, enconrtrar el equivalente de Norton entre los terminales A-B.

\begin{figure}[H]
	\centering
	\includegraphics[scale=0.8]{img/6.png}
	\label{e8}
\end{figure}

Rta: $\bar{I}_N=2,43\angle{210,68°}[A]$; $Z_N=5,71\angle{-47,77°}\Omega$
% ***********************************************
% 					Ejercicio 9
% ***********************************************
\item Obtener el quivalente de Thévenin del siguiente circuito entre los terminales A-B. 

\begin{figure}[H]
	\centering
	\includegraphics[scale=0.8]{img/7.png}
	\label{e9}
\end{figure}

Rta: $V_{Th}=20V$ $Z_{Th}=1,22\Omega$
% ***********************************************
% 					Ejercicio 10
% ***********************************************

\item Encontrar el equivalente de Thévenin del siguiente circuito entre los términales A-B. Resolver Alplicando el método de corrientes de mallas.

\begin{figure}[H]
	\centering
	\includegraphics[scale=0.8]{img/8.png}
	\label{e10}
\end{figure}

Rta: $\bar{V}_{Th}=20\angle{0°}[V]$; $Z_{Th}=(3,32+j1,41)\Omega$

% ***********************************************
% 					Ejercicio 11
% ***********************************************
\item  Dado el siguiente circuito, encontrar el equivalente de Thévenin entre los terminales A-B

\begin{figure}[H]
	\centering
	\includegraphics[scale=0.8]{img/9.png}
	\label{e11}
\end{figure}


Rta: $\bar{V}_{Th}=835,22 \angle{-20,17°}[V]$; $Z_{Th}=(120-j60)\Omega$

\section{Resonancia y lugares geométricos.}
% ***********************************************
% 					Ejercicio 12
% ***********************************************

\item Dado el siguiente circuito, calcular:

\begin{enumerate}
	\item La frecuencia de resonancia.
	\item El factor de mértito Q.
	\item El ancho de banda.
	\item Las frecuencias cuadrantales.
	\item La tensión en cada elemento cuando el circuito se encuentra en resonancia.
	
\end{enumerate}

\begin{figure}[H]
	\centering
	\includegraphics[scale=0.8]{imgb/1.png}
	\label{e12}
\end{figure}

a) $\omega_0=4000 \frac{rad}{seg}$ ; b) $Q=2$; c) $BW=2000\frac{rad}{seg}$; d) $f_1=497Hz$; $f_2=815,36$; e) $\bar{V}_R=100\angle{0°}[V]$; $\bar{V}_L=200\angle{90°}[V]$; $\bar{V}_C=200\angle{-90°}[V]$

% ***********************************************
% 					Ejercicio 13
% ***********************************************
\item En el circuito del ejercicio 1, qué elemento habrá que modificar para aumentar el ancho de banda a $3000 \frac{rad}{seg}$, sin alterar $\omega_0$. Justifique. ¿Cuál sería su nuevo valor?

Rta: $R=15\Omega$

% ***********************************************
% 					Ejercicio 14
% ***********************************************
\item Se desea disminuir en un $50\%$ el ancho de banda del siguiente circuito manteniendo constante $\omega_{0}$ e $\bar{I}$. ¿Qué elementos se deberán modificar y cuáles son sus valores?

\begin{figure}[H]
	\centering
	\includegraphics[scale=0.8]{imgb/2.png}
	\label{e14}
\end{figure}

Rta: $L=20mHy$; $C=12,5\mu F$
% ***********************************************
% 					Ejercicio 15
% ***********************************************
\item Calcular el valor de $R_L$ para que el circuito de la figura entre en resonancia.

\begin{figure}[H]
	\centering
	\includegraphics[scale=0.8]{imgb/3.png}
	\label{e15}
\end{figure}


% ***********************************************
% 					Ejercicio 16
% ***********************************************
\item La tensión aplicada a un circuito serie formado por una reactancia inductiva fija $X_L=5\Omega$ y una resistencia varible $R$, es $\bar{V}=50\angle{0^{\circ}}[V]$. Trazar los lugares geométricos de la admitancia y de la intensidad de corriente.

% ***********************************************
% 					Ejercicio 17
% ***********************************************
\item En el circuito paralelo de la figura, la reactancia inductiva puede variar sin límites. Trazar el lugar geométrico de la admitancia total y demostrar que no es posible la resonancia.

\begin{figure}[H]
	\centering
	\includegraphics[scale=0.8]{imgb/4.png}
	\label{e17}
\end{figure}
% ***********************************************
% 					Ejercicio  18
% ***********************************************
\item  Hallar el valor de $R$ para que el siguiente circuito entre en resonancia y trazar el lugar geométrico de la admitancia equivalente del circuito.

\begin{figure}[H]
	\centering
	\includegraphics[scale=0.8]{imgb/5.png}
	\label{e18}
\end{figure}


    
\end{enumerate}
    
\end{document}
